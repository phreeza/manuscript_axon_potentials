\documentclass[]{elife}

\title{Dipolar~extracellular~potentials generated~by~axonal~projections}
\author[1*]{Thomas McColgan}
\author[2]{Ji Liu}
\author[1,3]{Paula T. Kuokkanen}
\author[2]{Catherine E. Carr}
\author[4]{Hermann Wagner}
\author[1,3,5*]{Richard Kempter}
\affil[1]{Department of Biology, Institute of Theoretical Biology, Humboldt-Universität zu Berlin, Berlin, Germany}
\affil[2]{Department of Biology, University of Maryland, College Park, MD, USA}
\affil[3]{Bernstein Center for Computational Neuroscience, Berlin, Germany}
\affil[4]{Institute for Biology II, RWTH Aachen, Aachen, Germany}
\affil[5]{Einstein Center for Neurosciences, Berlin, Germany}

\corr{thomas.mccolgan@gmail.com}{TM}
\corr{r.kempter@biologie.hu-berlin.de}{RK}
%\author{Thomas
%McColgan,\footnote{Department of Biology, Institute of Theoretical Biology, Humboldt-Universität zu Berlin, Berlin, Germany} \and Ji
%Liu,\footnote{Department of Biology, University of Maryland, College Park, MD, USA} \and Paula T Kuokkanen,\footnotemark[1] \footnotemark[2] \and Catherine E Carr,\footnotemark[2] \and Hermann
%Wagner,\footnote{Institute for Biology II, RWTH Aachen, Aachen, Germany} \and Richard
%Kempter\footnotemark[1] \footnote{Bernstein Center for Computational Neuroscience, Berlin, Germany}
%\footnote{Einstein Center for Neurosciences, Berlin, Germany}}
\date{}

\begin{document}
\maketitle
\begin{abstract}
Extracellular field potentials (EFPs) are an important source of
information in neuroscience, but their physiological basis is in many
cases still a matter of debate. Axonal sources are typically discounted
in modeling and data analysis because their contributions are assumed to
be negligible. Here, we established experimentally and theoretically
that contributions of axons to EFPs can be significant. Modeling action
potentials propagating along axons, we showed that EFPs were prominent
in the presence of terminal zones where axons branch and terminate in
close succession, as found in many brain regions. Our models predicted a
dipolar far field and a polarity reversal at the center of the terminal
zone. We confirmed these predictions using EFPs from the barn owl
auditory brainstem where we recorded in nucleus laminaris using a
multielectrode array. These results demonstrate that axonal terminal
zones can produce EFPs with considerable amplitude and spatial reach.
\end{abstract}

\section{Introduction}\label{introduction}

Extracellular field potentials (EFPs) are at the heart of many
experimental approaches used to examine the inner workings of the brain.
Types of EFPs include invasively recorded signals such as the
electrocorticogram (ECoG) and the local field potential (LFP), as well
as the noninvasively recorded electroencephalogram (EEG) and the
auditory brainstem response (ABR)
\citep{Brette2012Handbook, Nunez2006Electric}. Measures derived from the
EFP such as the current source density (CSD) and multiunit activity
(MUA) are also frequently used. The origins of these signals and
measures, especially in cases in which the activity is not clearly
attributable to a single cell, is a matter of debate
\citep{Buzsaki2012Origin}.

EFPs in the brain were long thought to be primarily of synaptic origin
\citep{Buzsaki2012Origin}. As a consequence, many modeling studies
focussed on the extracellular fields induced by postsynaptic currents on
the dendrites and the soma of a neuron
\citep{Holt1999Electrical, Einevoll2013Modelling, Linden2011Modeling, Linden2010Intrinsic, FernandezRuiz2013Cytoarchitectonic}.
However, a number of recent data analyses and modeling efforts have
revealed that active, non-synaptic membrane currents can play an
important role in generating population-level EFPs
\citep{Reimann2013Biophysically, Anastassiou2015Cell, Schomburg2012Spiking, Ray2011Different, Belluscio2012CrossFrequency, Waldert2013Influence, Ness2016Active, SchefferTeixeira2013HighFrequency, Reichinnek2010Field, Sinha2015HCN, Taxidis2015Local},
including far reaching potentials detectable at the scalp
\citep{Telenczuk2011Highfrequency, Telenczuk2015Correlates}. Currents
from the axon are still thought to be so small as to be of minor
importance for the EFP.

One of the reasons for the assumption that axonal currents contribute
little to the EFP is that the far field of an action potential traveling
along an idealized straight and long axon is quadrupolar, meaning that
it decays faster with distance than synaptic sources, which are
typically dipolar \citep{Nunez2006Electric}. Surprisingly, theoretical
\citep{Tenke1993Interpretation} and experimental studies indicated that
the EFP of axonal responses may also have a dipolar structure. For
example, \citet{Blot2014Ultrarapid} reported an EFP with a
characteristic dipolar structure in the vicinity of cerebellar Purkinje
cell axons; other studies
\citep{Swadlow2002Activation, Swadlow2000Influence} showed that the
axonal part of the EFP of thalamocortical afferents showed a polarity
reversal associated with a dipolar source, and classical current source
density studies found dipolar current distributions in axonal terminal
zones in the cortex and the lateral geniculate nucleus, and attributed
these to axons because of conduction velocities
\citetext{\citealp{Mitzdorf1978Prominent}; \citealp{Mitzdorf1985Current}; \citealp{Mitzdorf1977Laminar}; \citealp[see
also][]{Tenke1993Interpretation}}. Here we introduce another
experimental system and show a strong dipolar, axonal field potential in
the auditory brainstem of the barn owl.

The discrepancy between the quadrupolar structure of EFPs generated by
idealized axons, and the experimentally observed dipolar structure
raises the question of how axons are able to generate dipolar field
potentials. In this manuscript we show how dipolar far fields in the EFP
of axons can be explained by the axons' anatomical structure. In
particular, the branchings and terminations of axons in their terminal
zone area deform the extracellular waveform
\citep{Gydikov1986Influence, Gydikov1986Extracellular, Plonsey1977Action}
and can lead to a dipolar EFP structure. Axon bundles, sometimes called
fascicles, exist throughout the peripheral and central nervous system
and have such terminal zones
\citep{kandel2000principles, Hentschel1999Models, Nornes1972Temporal, Goodman1984Cell}.
The white matter of the mammalian brain can be viewed as an
agglomeration of such fascicles \citep{Schuz2002Human}. We therefore
predict pronounced contributions of axon bundles to EFPs, which are
neglected in current models.

In what follows, axonal contributions to the EFP are first investigated
by a numerical model based on forward simulation
\citep{Holt1999Electrical, Gold2006Origin}. This first model includes a
large-scale multi-compartment simulation
\citep{Jack75Electric, Rall1959Branching, Abbott1992Simple, Hines1997NEURON, Hines2009NEURON}
of an axon population. We then outline the basic mechanisms by means of
a second, analytically tractable, model of a generic axon bundle.
Finally, we validate model predictions with data from multi-site in-vivo
electrophysiological recordings from the barn owl auditory brainstem.

\section{Results}\label{results}

\subsection{Effects of axonal bifurcations and terminations on
extracellular action
potentials}\label{effects-of-axonal-bifurcations-and-terminations-on-extracellular-action-potentials}

To understand how the geometry of an axon affects the extracellular
waveform associated with action potentials, we first numerically
simulated single action potentials propagating along generic axons and
calculated their contribution to the EFP (for details, see Materials and
Methods). This was done for five scenarios: quasi-infinite axons,
terminating axons, bifurcating axons, axons that bifurcate as well as
terminate, and axon bundles (Figure~\ref{fig:simpletree}). We began by
simulating a long axon, approximating an infinitely long axon following
a straight line path, neither bifurcating nor terminating
(Figure~\ref{fig:simpletree}A). The extracellular action potential has
the characteristic triphasic shape. As the action potential travels
along the axon, the waveform is translated in time with the conduction
velocity, but is otherwise unchanged. The triphasic shape is also
present in the spatial arrangement of transmembrane currents at any
given time, which is the reason for the quadrupolar EFP response
traditionally assumed for axons.

\begin{figure}[htbp]
\centering
\includegraphics{figs/fig_1_nocsd.pdf}
\caption{\label{fig:simpletree}Relationship between axon morphology and
extracellular potential. Multi-compartment simulations of action
potentials traveling along axons with varying morphologies, as indicated
by the diagram on the left-hand side of each subfigure. Action potential
propagation direction indicated by arrow. Waveforms, shown on the
right-hand side of each subfigure, were recorded at a horizontal
distance of 150~µm from the axons. The vertical depth is indicated by
the plot position, spaced by 400~µm. Horizontal plot location and
distances between axons are for illustration only, all axons were
simulated to lie on a straight line. (\textbf{A}) Action potential in a
quasi-infinitely long, straight axon. (\textbf{B}) Terminating axon.
Action potential waveform closest to the termination thickened for
emphasis. (\textbf{C}) Branching axon. The axon branches multiple times
within of 200~µm. Thicker waveform at the center of the bifurcation
zone. (\textbf{D}) Combined bifurcations and terminations. Note the
larger voltage scales in C and D, which correspond to the different
number of fibers. (\textbf{E}) Response in a population of 100
randomized morphologies, three of which are shown schematically
(colored). Activity consists of spontaneous background activity (100
spikes/s) superimposed with a brief Gaussian pulse of heightened spike
rate (2000 spikes/s). Spike rate and example spike times for the three
morphologies are shown at the top. Right: gray lines show activity of
full population averaged over 40 trials, while the black lines show the
low-pass (\textless{}1 kHz) component. Note that the time and voltage
scales are different from A-D. In all graphs, spatial scales are the
same, as indicated by the scale bar in A.}
\end{figure}

There are two ways of understanding the triphasic shape of the
extracellular waveform. One way is by attribution of the peaks of the
response to individual current types. The first, small positive peak
corresponds to the capacitive current, the large and negative second
peak to the sodium current, and the final positive peak to the potassium
current \citep{Gold2006Origin}. Another, more mathematical way of
understanding the triphasic shape is specific to the nature of the axon.
Due to Kirchhoff's law and cable theory (see Materials and Methods), the
local transmembrane current in a homogeneous axon is proportional to the
second spatial derivative of the membrane potential along the direction
of the axon. Because the action potential is roughly a traveling wave,
the currents are also proportional to the second \emph{temporal}
derivative of the membrane potential. The three extrema of the EFP are
thus related to the points of maximum curvature in the action potential
waveform, namely the onset, the maximum, and the end of the spike.

Next we simulated the response of an axon that terminates
(Figure~\ref{fig:simpletree}B). Here the action potential approaching
the recording location (top traces) has the same, triphasic EFP response
as in the non-branching case. When the action potential reaches the
termination point, its EFP gradually deforms into a biphasic response,
with a positive peak preceding a negative peak. The mechanism for this
deformation can be understood as follows: As the action potential
approaches the recording location next to the termination, the majority
of the transmembrane currents flow at points located before the
termination, and they are almost identical to those in the
non-terminating case; the first, capacitive peak is not affected. As
already mentioned, the second and third peaks of the extracellular
action potential in the non-terminating case are generated by currents
close to or after the electrode location. In the terminating axon, there
are no currents at points after the termination, leading to a partial
suppression of the second peak and a complete suppression of the third
peak.

Another generic structure found in axons is a bifurcation. To emphasize
the impact of bifurcations, we simulated a single axon that bifurcates
three times on each branch within a distance of 200~µm (100~µm between
branchings), leading to a total number of 8 collaterals leaving the
bifurcation zone (Figure~\ref{fig:simpletree}C). (Note that in order to
avoid confounding effects, the horizontal distances between axons in
Figure~\ref{fig:simpletree}C-E are for illustration only; all
collaterals were simulated to lie on a straight line.) The EFP far away
from the bifurcation zone has a triphasic shape and resembles the one
observed in Figure~\ref{fig:simpletree}A, and the amplitude is
proportional to the number of axon fibers. The EFP near the bifurcation
zone has a biphasic shape. Although there is an initial tiny positive
peak, the response is dominated by the second, negative and the third,
positive peak. This waveform can again be understood by comparison to
the first example (Figure~\ref{fig:simpletree}A) containing the
infinitely long axon: The tiny positive initial peak resembles the
infinite case, because it is constituted by the action potential-related
currents flowing within the part of the axon before the bifurcation. As
the action potential passes the bifurcation zone, there are now several
action potentials (one in each fiber). Because of the active nature of
the action potential, the active currents are the same in each outgoing
fiber as in the incoming fiber. This leads the second and third peak to
be multiplied in size, yielding a quasi-biphasic response. We chose to
simulate several bifurcations because this leads to a clearer effect in
the EFP. In the case of a single bifurcation, this effect is also
present, but the amplification of the second and third peak relative to
the first peak is not as notable as in this example.

To understand how bifurcations and terminations interact when they are
present in the same axon, we simulated an axon with an identical number
of bifurcations as in the previous case, but then added terminations to
all the fibers 700~µm after the bifurcation zone
(Figure~\ref{fig:simpletree}D). We found that this configuration leads
to the same biphasic responses as observed in the cases of the isolated
anatomical features. A triphasic response occurred in-between the
bifurcation and termination zones. A notable point here is that the
potential at the bifurcation and termination are both biphasic and on
the same timescale, but opposite in polarity.

After having studied the EFPs of single axons, we next started to
simulate axon bundles, because axons often run in parallel bundles in
the brain. Moving towards more biologically plausible axon geometries,
we considered bundles consisting of axons with slightly altered spatial
arrangement: we randomly perturbed the precise locations of bifurcations
and terminations in the axon tree (Figure~\ref{fig:simpletree}E, left;
for details, see Materials and Methods). We simulated 100 axons and
stimulated each axon with an inhomogeneous Poisson spike train
\citep{Kuokkanen2010Origin, Softky1993Highly}. The driving rate of the
inhomogeneous Poisson process was the same for all axons and consisted
of a constant background rate (100~spikes/s) and a Gaussian pulse of
heightened activity (2000~spikes/s). The standard deviation of the pulse
was 1 ms, resulting in an additional 3.5 spikes per axon being fired on
average over the entire duration of the pulse. The resulting
extracellular population-level waveforms (Figure~\ref{fig:simpletree}E,
right) showed a polarity reversal reminiscent of
Figure~\ref{fig:simpletree}D. However, in the bifurcation zone, the
summed contribution from many fibers and action potentials lead to a
monophasic negative peak, and in the termination zone there was a
monophasic positive peak. Interestingly, the summed potential at the
center of the terminal zone largely cancelled out.

The fact that the responses in Figure~\ref{fig:simpletree}E were mostly
monophasic can be explained by the presence of a non-zero bias in the
biphasic responses observed for the single spike responses in
Figure~\ref{fig:simpletree}D: close to a bifurcation, the area under the
negative part of the curve slightly exceeded that of the positive part,
and vice versa close to a termination. When summed up over many spikes
with different timings, this difference in areas induced a positive or
negative polarity of the population response in
Figure~\ref{fig:simpletree}E.

The reversal behaviour shown in Figure~\ref{fig:simpletree}E is similar
to the polarity reversal associated with a dipole observed in
experimental studies
\citep{Swadlow2000Influence, Swadlow2002Activation, Blot2014Ultrarapid}.
To summarize, simple one-dimensional model axon structures can produce
complex and diverse spatiotemporal EFP responses, including monophasic,
biphasic and triphasic waveforms, comparable to experimentally recorded
responses.

\subsection{Axonal projections generate a dipole-like field
potential}\label{axonal-projections-generate-a-dipole-like-field-potential}

Dipole-like EFPs have a much larger spatial reach than quadrupolar-like
EFPs, which are typically associated with axons
\citep{Nunez2006Electric}. To further understand whether and how axons
can generate a dipolar EFP, in Figure~\ref{fig:bigtree} we turned to
three-dimensional axon morphologies, in contrast to the one-dimensional
case studied in Figure~\ref{fig:simpletree} (for details, see Materials
and Methods). We thus simulated a parallel fiber bundle of 5000 axons
that at first runs at a constant number of fibers without bifurcations
and then reaches a terminal zone. Within this terminal zone, the fibers
first bifurcate, which increases the number of fibers. Finally, as the
axons reach the end of the terminal zone, they terminate and the number
of fibers decreases to zero (example axon shown in
Figure~\ref{fig:bigtree}A). To model more closely the actual axonal
structures found in nature, we included a radial fanning out of the
branches as well as a more diverse set of morphologies with a variable
number of bifurcations and terminations (for details, see Materials and
Methods).

The spiking activity of a generic axon bundle was simulated by a
background spontaneous firing rate of 100 spikes/s and a short pulse of
increased activity. We chose a Gaussian pulse with an maximum
instantaneous rate of 2900 spikes/s and a standard deviation of 2.8~ms.
Note that this high driving rate is only the instantaneous maximum, and
the actual firing rate is limited by the refractory period following a
spike. These numbers are motivated by the early auditory system of barn
owls \citep{koppl97a, Sullivan1984Segregation, Konishi1985Owls}, where
instantaneous spike rates of 3000 spikes/s occur in response to click
stimuli \citep{Carr2016Role}. However, our approach is not limited to
the auditory system (which would also require the introduction of the
synchronization of the spike times to the auditory stimulation
frequency, called phase locking). Instead, this pulse of activity could
relate to various kinds of evoked activity in the nervous system, such
as sensory stimulation, motor activity or a spontaneous transient
increase in population spiking activity.

To characterize the spatiotemporal dynamics of the evoked EFP, the time
course of the potential was calculated for several locations along the
axon trunk. The responses were averaged over 10 repetitions. We divided
our analysis into two frequency bands by filtering the responses. The
first frequency band was obtained by low-pass filtering with a cutoff
frequency of 1~kHz (Figure~\ref{fig:bigtree}A--C). The second frequency
band was the multiunit activity (MUA) obtained by high-pass filtering
with a cutoff frequency of 2.5~kHz (Figure~\ref{fig:bigtree}D--F). To
make the MUA easier to interpret in terms of overall activity reflected,
it was half-wave rectified and low-pass filtered (\textless{}500~Hz, see
Materials and Methods). The two frequency bands showed a qualitatively
different spatiotemporal response in the vicinity of the projection
zone, as we will show in the following.

We first studied the effect of the Gaussian rate pulse on the low-pass
filtered EFP (Figure~\ref{fig:bigtree}B). The filtering removed most of
the identifiable components of individual spikes, while a
population-level signal remained, similar to an LFP signal. The
distribution of the maximum amplitudes of these responses is shown by
the colored contour lines in Figure~\ref{fig:bigtree}A and the colored
voltage traces in Figure~\ref{fig:bigtree}B. Surrounding the terminal
zone of the axon bundle in Figure~\ref{fig:bigtree}A, low-pass filtered
EFP amplitudes showed a double-lobed shape typical of a dipole.

In Figure~\ref{fig:bigtree}B, the low-pass filtered EFP responses mostly
showed monophasic deflections elicited by the population firing rate
pulse, in a manner similar to that observed in
Figure~\ref{fig:simpletree}E. Such deflections were visible at all
recording locations. In the radial direction away from the axon tree,
i.e.~in the horizontal direction in the figure, the low-pass filtered
EFP amplitude decays. In the axial direction along the axon tree,
i.e.~in vertical direction in the figure, the voltage deflection
reverses polarity in the middle of the terminal zone of the bundle
(Figure~\ref{fig:bigtree}B). The polarity reversal occurs by a decrease
of the amplitude to zero and a subsequent reappearance with reversed
polarity (as opposed to a polarity reversal through a gradual shift in
phase). This behaviour is also typical for a dipolar field potential.

The point of the polarity reversal coincides with the middle of the
terminal zone. Interestingly, this means that the absolute value of the
response amplitude reaches a local minimum just at the axial location at
which the number of axonal fibers reaches a maximum. To better
understand how the anatomical features of the axon bundle and the EFP
response amplitude are related, we compared its signed maximum value
(meaning the \emph{signed} value corresponding to the maximum
\emph{magnitude} of the amplitude of the EFP, black line in
Figure~\ref{fig:bigtree}C) with the change of the number of nodes per
200~µm bin (purple histogram in Figure~\ref{fig:bigtree}C): Along the
nerve trunk the number of fibers is constant. As the axon bundle reaches
its terminal zone, the number of bifurcations increases (purple bars
point to the right in Figure~\ref{fig:bigtree}C). The increase of
bifurcations is followed by an increase in terminations. In the middle
of the terminal zone, the number of bifurcations and terminations are
equal. At the same depth, the amplitude of the EFP component crosses
zero. At the end of the terminal zone, the terminations outweigh the
bifurcations (purple bars point leftwards in Figure~\ref{fig:bigtree}C).
As the axon bundle ends, there are no longer any bifurcations or
terminations, and the number of fibers decays toward zero. Overall, the
signed maximum amplitude EFP (black trace) follows the distribution of
branchings and terminations (purple histogram). This progression of
amplitudes in the low-frequency components seen in
Figure~\ref{fig:bigtree}C is also visible in Figure~\ref{fig:bigtree}B,
most clearly in the first column. The polarity reversal in the center of
Figure~\ref{fig:bigtree}B corresponds to the crossing of zero amplitude
in Figure~\ref{fig:bigtree}C.

\begin{figure}[htbp]
\centering
\includegraphics[height=0.65\vsize]{figs/fig_2.pdf}
\caption{\label{fig:bigtree}An activity pulse in an axonal projection
generates a dipole-like extracellular field potential (EFP).
(\textbf{A}) Modeled example axon from the simulated bundle in black,
along with iso-potential lines for the low-pass filtered
(\textless{}1~kHz) EFP signature of the activity pulse. The contours
(amplitudes in mV as indicated by colorbar) show the typical double-lobe
of a dipole. (\textbf{B}) The low-pass filtered EFP waveforms, recorded
at the locations of the colored dots in \emph{A}, show mostly unimodal
peaks. The peak amplitude reverses polarity as a function of recording
location in the vertical direction. The reversal occurs by inverting the
amplitude with approximately unchanged shape. (\textbf{C}) Progression
of the maximum low-pass filtered EFP amplitude with depth (black line)
at a distance of 100~µm from the trunk (indicated by arrow in A). The
amplitude closely follows the local change (spatial derivative) in
number of nodes per unit length (purple histogram), which is
proportional to the difference in number between bifurcations and
terminations. (\textbf{D}) Modeled axon from bundle as in A, and
iso-potential contours for the multi-unit activity (MUA) component.
(\textbf{E}) Response waveforms of the MUA component. High-pass filtered
(\textgreater{}2.5~kHz) component (the first processing stage for
calculation of MUA, see Materials and Methods) in black. (\textbf{F})
Maximum amplitude of the MUA component (black line) follows the number
of fibers (teal-colored histogram). Note the different units of the
histograms in (C) and (F), due to the fact that (C) is the derivative in
space of (F).}
\end{figure}

To understand how the EFP contributions are related to individual
spikes, we next turned our attention to the high-frequency MUA. The MUA
is thought to reflect local spiking activity
\citep{Stark2007Predicting}. In Figure~\ref{fig:bigtree}D, the
iso-amplitude lines of the MUA appeared like an ellipsoid centered at
the terminal zone (Figure~\ref{fig:bigtree}D); they did not show the
double-lobe shape observed for the low-pass filtered EFP in
Figure~\ref{fig:bigtree}A.

The shape of the MUA response was weakly dependent on the recording
location. The main change across locations was in the scaling of the
amplitude (Figure~\ref{fig:bigtree}E). The amplitude decays with radial
distance from the trunk. In the axial direction, the amplitude reaches
its maximum in the middle of the fiber bundle. This dependence of the
MUA amplitude on the axial location is further examined in
Figure~\ref{fig:bigtree}F. The amplitude of the MUA component (black
trace) changes in accordance with the local number of nodes per unit
length (teal-colored histogram), which is proportional to the number of
fibers. The local number of fibers and the MUA amplitude are both
constant along the nerve trunk. Both measures then increase in amplitude
as the number of fibers is increased by bifurcations. As the fibers
terminate and the number of fibers decreases, so does the amplitude of
the MUA.

To conclude, we have shown a qualitatively different behaviour in the
low- and high-frequency components of the EFP, i.e.~for the low-pass
filtered EFP and the MUA. The particular branching and terminating
structure of the axon bundle may thus give rise to a dipolar low-pass
filtered EFP.

\subsection{Effects of bifurcations and terminations on distance scaling
of
EFPs}\label{effects-of-bifurcations-and-terminations-on-distance-scaling-of-efps}

To further demonstrate that bifurcations and terminations of axons give
rise to a dipolar field, we investigated the effect of an axon terminal
structure on the spatial reach of the EFP
(Figure~\ref{fig:distscaling}). Motivated by the fundamentally different
spatial distributions of the low-pass filtered EFP and the
high-frequency MUA in Figure~\ref{fig:bigtree}, we again differentiated
between these frequency bands and simulated an axon bundle containing a
terminal zone with bifurcations and terminations. Moreover, as a
control, we also simulated an axon bundle without bifurcations in which
a fixed number of fibers simply terminates.

\begin{figure}[htbp]
\centering
\includegraphics[height=0.65\vsize]{figs/fig_3.pdf}
\caption{\label{fig:distscaling}The low-frequency (\textless{}1~kHz)
component of the axon bundle EFP is influenced supralinearly by a
projection zone, while the high-frequency (\textgreater{}2.5~kHz)
component is not. (\textbf{A}) Scaling of the low-frequency
(\textless{}1~kHz) EFP component. (\emph{Top}) The spatial wavelength of
the membrane potential oscillation (blue) is larger than the width of
the projection zone (gray). (\emph{Bottom}) The amplitude of the
low-frequency EFP component for the bifurcating case (solid line) decays
with with axial distance from the axon bundle. It always always exceeds
the EFP amplitude of the non-bifurcating case (dashed line). Note the
double-logarithmic scale. Axial distances $r$ are calculated from the
center of the terminal zone. For comparison, scaling that follows
$r^{-2}$ is indicated with the black line. (\textbf{B}) Same as A but
for the high-frequency (\textgreater{}2.5~kHz) EFP component.
(\emph{Top}) The spatial wavelength of the membrane potential
oscillation (green) is much smaller than the width of the projection
zone (gray). (\emph{Bottom}) The amplitude of the high-frequency EFP
decays several orders of magnitude within the terminal zone, and the
amplitude is larger in the bifurcating case (solid line) compared to the
non-bifurcating case (dashed line). Far away from the terminal zone,
i.e., for axial distances $r> 1$~mm, they decay proportional to
$r^{-2}$ but with similar amplitudes.\\
(\textbf{C}) Normalized dipole moments of the bifurcating and
non-bifurcating bundles as a function of frequency. (\textbf{D}) Ratio
of the dipole moments between bifurcating and non-bifurcating cases (red
line), compared to the maximum ratio 10 of the number of fibers (dotted
line), to indicate supralinear (\textgreater{}10) and sublinear
(\textless{}10) contributions. Vertical gray line in C and D indicates
the width (\textasciitilde{}2 mm) of the projection zone.}
\end{figure}

In order to separate the effects of any radial fanning out of the axon
bundle from the effects of bifurcations and terminations, and to afford
better analytic tractability, we transitioned back to a simpler
one-dimensional model of the axon bundle (see Materials and Methods).
This model omitted the radial fanning out of the bundle in the terminal
zone, as in Figure~\ref{fig:simpletree}. Furthermore, we discarded the
detailed conductance-based simulation of the membrane potential, and
instead assumed a fixed membrane potential waveform traveling along the
axon trunk with a constant propagation velocity. Using linear cable
theory, it was then possible to calculate the membrane currents
necessary for the determination of the EFP. The analytic nature of the
simplified model also allowed us to consider a continuous number of
fibers instead of simulating discrete bifurcations and terminations. All
following simulations are based on this simplified model.

To verify that this simpler one-dimensional model accurately captures
the EFP response of an axon bundle, we applied parameters equivalent to
those used in Figure~\ref{fig:bigtree} and compared the resulting EFP to
that obtained from the full biophysical model. We calculated the
relative difference of the EFPs by taking the absolute value of their
differences, and normalizing by the sum of their absolute values.
Averaged over time, this relative difference at distances greater than 1
mm from the center of the projection zone was \textless{} 0.05 in axial
direction. In the radial direction, we found, as expected, larger
relative discrepancies of \textless{} 0.3 for radial distances
\textgreater{} 1~mm. In what follows, we focus on the axial direction,
which is the dipole axis.

Let us now specify how we simulated the two axon bundle morphologies.
The control case was a non-bifurcating bundle, which had a constant
number of 50 fibers up to the termination point, and then tapered out
with a Gaussian profile that was centered at the termination point with
a height of 50 fibers and width (standard deviation) of 300~µm. The
second case was that of an axon bundle with a projection zone containing
bifurcations. Here we added to the distribution of the number of axons
used for the non-bifurcating control case a further Gaussian
distribution to account for the projection zone. This additional
Gaussian was also centered at the termination point, but had an
amplitude of 450 fibers and a standard deviation of 500~µm, meaning that
the overall width of the terminal zone was \(\approx\) 2~mm. Unlike the
tapering-out in the control condition, this component was added for both
before and after the termination point. It resulted in a maximal fiber
number of 500 at the termination point, which is a factor 10 larger than
in the control case. Both distributions constructed in this way were
smooth, and they had smooth first derivatives in space. In both cases,
the number of fibers decreased monotonically after the termination
point. We considered a conduction velocity of 1 m/s in this example,
though results are qualitatively the same for other values. In order to
understand the frequency-specific effects of the projection zone, we
calculated the responses to membrane potential components with temporal
frequencies between 25~Hz and 5~kHz, with the same amplitude for each
frequency component. For the conduction velocity 1~m/s, these temporal
frequencies corresponded to spatial wavelengths from 10~mm to 0.2~mm,
i.e.~from much larger to much smaller than the width of the terminal
zone. We then calculated for each frequency/wavelength the average
amplitude of the resulting EFP response by taking its standard
deviation. Due to the linear nature of our model, the frequency
responses obtained in this manner are applicable to the Fourier
components of any membrane potential time-course.

The dipole-like component observed in Figure~\ref{fig:bigtree} for the
low-frequency component had its dipole axis aligned with the axon trunk.
We therefore considered the distance \(r\) beyond the termination point
in the direction extending the axon trunk, which we called the axial
direction. Because we suspected a dipolar response, we expected the
amplitude of the field potential to decay as \(r^{-2}\). To test the
scaling behaviour of this component, we first plotted the average
amplitude of the low-frequency responses (\(f\)\textless{}1~kHz) in
axial direction in Figure~\ref{fig:distscaling}A. The plot is on a
double logarithmic scale, meaning that the slope of the curve
corresponds to the scaling exponent, and the vertical offset corresponds
to the amplitude of the size of the dipole moment, which is a measure
for the strength of the dipolar EFP. We observed the expected \(r^{-2}\)
scaling for distances larger than the extent of the bifurcation zone
(\(\gtrsim 1\)~mm).

Comparing the responses of the bifurcating axon bundle and the
non-bifurcating control condition (full and dashed lines in
Figure~\ref{fig:distscaling}A), we saw that for short distances
(\textless{} 1~mm) the response of the bifurcating case was a factor 10
larger than the control. At these distances the response was due to the
local fibers, of which there are 10 times more in the bifurcating case.
At distances larger than 1~mm, we observed that the distance scaling was
proportional to \(r^{-2}\), meaning that there was a dipole moment in
both conditions (a vanishing dipole moment would have implied a slope
steeper than \(r^{-2}\)). Interestingly, for distances larger than 1~mm
the response in the bifurcating case exceeded the control by a factor
20. We thus concluded that at low frequencies, the bifurcation zone
contributes supralinearly to the dipole moment.

The reason for this supralinearity is that contributions from different
parts of the axon bundle can interfere constructively or destructively.
The maximum constructive interference occurs when the spatial width of
the oscillation agrees with that of the projection zone
(Figure~\ref{fig:distscaling}A, top). Importantly, currents from fibers
inside the projection zone on average interact destructively with those
from outside the projection zone. The magnitude of this effect depends
on the ratio of the number of fibers inside the projection zone compared
to the number outside. The larger the ratio the smaller the impact of
destructive interference. Thus, bifurcations suppress the destructive
interference (Figure~\ref{fig:distscaling}A, bottom, full line). On the
other hand, for a ratio of one, i.e.~the non-bifurcating case,
destructive interference is strong, which diminishes the overall
response amplitude (Figure~\ref{fig:distscaling}A, bottom, dashed line).

Next, we examined the high-frequency (\textgreater{}2.5~kHz) component
(Figure~\ref{fig:distscaling}B). As in the low-frequency case, the
response at distances \textless{} 1~mm was greater in the bifurcating
case by a factor of 10. The asymptotic scaling was also \(r^{-2}\) for
axial distances \(> 1\)~mm in both cases. However, unlike in the
low-frequency case, the amplitudes were similar between bifurcating and
non-bifurcating cases. Thus, the presence of a bifurcation zone did not
contribute to the high-frequency dipole moments in the EFP. This feature
is explained by the small spatial wavelength of the stimulus compared to
the width of the bifurcation zone (Figure~\ref{fig:distscaling}B, top)

\subsection{Frequency-dependence of dipolar axonal
EFPs}\label{frequency-dependence-of-dipolar-axonal-efps}

We showed that the dipole moment depends on both the anatomy, i.e.~the
presence of a projection zone, and the temporal frequency range (low
vs.~high frequencies) of the underlying activity. This relationship can
be qualitatively understood by considering that in an axon bundle a
voltage waveform propagates at some conduction velocity. The temporal
frequency of this signal thus corresponds to a spatial frequency. If the
spatial frequency of the membrane potential matches the width of the
projection zone, the dipole moment can reach its maximum. In this case,
at some point in time, positive membrane currents flow in one half of
the projection zone and negative membrane currents flow in the other
half (Figure~\ref{fig:distscaling}A, top). For example, if the voltage
waveform has a temporal frequency of 1~kHz and propagates at 1~m/s, the
spatial wavelength is 1~mm. If the spatial width of the termination zone
is about 1~mm, the dipole moment is maximal. In contrast, if the spatial
wavelength is much smaller than the width of the projection zone, an
alignment between projection zone and current flow is not possible, and
the dipole is not amplified (Figure~\ref{fig:distscaling}B, top) (for a
detailed derivation see Materials and Methods).

To quantitatively understand the frequency-specific contributions to the
dipole moments, we examined the scaling behaviour of the EFP as a
function of frequency. The amplitude of the dipole moment was determined
by fitting a straight line with slope \(-2\) to the double logarithmic
scaling of the standard deviation of the EFP at a given frequency. The
fit was performed for distances \(> 1\)~mm. The extrapolation of this
straight line to the axial distance 1~µm was then proportional to the
dipole moment.

The normalized frequency-specific dipole moments are shown in
Figure~\ref{fig:distscaling}C. The dipole moment of the bifurcating case
(solid line) has a maximum at around 500~Hz, as expected due to the
agreement of the spatial wavelength (1~m/s / 500~Hz = 2~mm; gray
vertical line in Figure~\ref{fig:distscaling}D and C) and axial width
(about 2~mm; gray box in Figure~\ref{fig:distscaling}A and B) of the
projection zone. The match is not exact because the shapes of sine wave
and Gaussian are different. For lower and higher frequencies, there is a
mismatch in spatial wavelength and the width of the projection zone,
meaning that the projection zone contributes only less to the dipole
moment, as also observed in Figure~\ref{fig:distscaling}B for higher
frequencies. In the non-bifurcating control case (dashed line) there is
no projection zone, and the dipole moment decays monotonically with
rising frequency because higher frequencies correspond to a smaller
spatial separation of positive and negative currents, and thus to a
smaller dipole moment.

In the bifurcating case, the maximum number of fibers was increased by a
factor of 10. Accordingly, an increase in the dipole moment by a factor
of 10 from the non-bifurcating to the bifurcating case could be
explained by just linearly summing the dipole moments of individual
fibers. An increase in the dipole moment by a factor greater than 10
would be supralinear. In Figure~\ref{fig:distscaling}D we compared this
relative impact of the terminal zone on dipole moments (red line) by
plotting the dipole moment ratios across frequencies. The contribution
of the terminal zone is greater than 10 (dotted line) for intermediate
frequencies between about 200 and 1300 Hz, and smaller than a factor 10
outside this frequency range.

Together, these observations show us that the terminal zone makes a
frequency specific contribution to the far-reaching dipole field
potential of the axon bundle. This provides a deeper understanding of
the findings of Figure~\ref{fig:bigtree}: At low frequencies
(\(<1\)~kHz), we observed a supralinear dipolar behaviour due to the
specific morphology the bundle, with the projection zone forming the
dipole axis. At higher frequencies, the bifurcation zone does not
amplify the dipole moment, meaning that we could observe responses
mainly locally.

\subsection{The barn owl neurophonic potential in nucleus laminaris as
an example for a dipolar field in an axonal terminal
zone.}\label{the-barn-owl-neurophonic-potential-in-nucleus-laminaris-as-an-example-for-a-dipolar-field-in-an-axonal-terminal-zone.}

To test our prediction of dipolar extracellular field potential
responses due to axon bundles, we recorded EFP responses from the barn
owl auditory brainstem. The barn owl has a highly developed auditory
system with a strong frequency-following response in the EFP (up to
9~kHz, \citet{Koppl1997b}), called the neurophonic, which can be
recorded in the nucleus laminaris (NL). In NL, the input from the two
ears is first integrated to calculate the azimuthal location of a sound
source, and this information is encoded in the EFP \citep{carr90}. The
EFP in this region is mainly due to the afferent activity, and the
contribution of postsynaptic NL spikes is small
\citep{Kuokkanen2010Origin, Kuokkanen2013Linear}. Furthermore, the
anatomy of the afferent axons is well known and follows a stereotypical
pattern \citep{carr88, carr90}: Two fiber bundles enter the nucleus,
with fibers from the contralateral ear entering ventrally, and from the
ipsilateral ear entering dorsally. The axon bundles reach the NL from
their origin without bifurcating, then bifurcate multiple times at the
border of the NL, and then terminate within NL. Axon bundles have a
strong directional preference and run roughly in parallel. Most of the
volume within NL consists of incoming axons. This well studied
physiology and anatomy makes the system an ideal candidate to
investigate the EFPs of axon bundles; see the Discussion for arguments
why synaptic contributions to the EFP could also be neglected here.

To explore the spatiotemporal structure of the EFP in NL, we performed
simultaneous multi-electrode recordings of the response in NL
(Figure~\ref{fig:expmethod}A) to contralateral monaural click stimuli.
The click responses showed distinct low-frequency
(Figure~\ref{fig:expmethod}B) and high-frequency
(Figure~\ref{fig:expmethod}C) components, as previously reported
\citep{wagner09}. The frequency of the high-frequency ringing
corresponds to the recording location on the frequency map within NL,
and the ringing reflects the frequency tuning and phase locking of the
incoming axons. In addition, there is a low-frequency component in the
response (Figure~\ref{fig:expmethod}B). We filtered the data to separate
these components, using the same cutoff frequencies as before for
low-frequency (\textless{}1 kHz) and high-frequency (\textgreater{}2.5
kHz) EFP.

The same simplified model used in Figure~\ref{fig:distscaling} was fit
to the data (example in Figure~\ref{fig:expmethod}) by performing a
nonlinear least squares optimization. The model considered only the
average membrane potential across the fibers, and we calculated the
membrane currents based on the density of fibers instead of simulating
individual fibers. The model also discarded the radial extent of the
bundle, treating it as a line; see Materials and Methods for more
details on the model. Free parameters to be fit were (1) the number of
fibers at the depth of each recording location, (2) the average spatial
derivative of the membrane potential over time in the fibers at the
location next to the most dorsal electrode, (3) the axonal conduction
velocity, and (4) the average distance between the axon bundle and
electrode array.

\begin{figure}[htbp]
\begin{fullwidth}
\includegraphics[width=0.95\linewidth]{figs/fig_4.pdf}
\caption{\label{fig:expmethod}Multielectrode recordings in the barn owl
show dipolar axonal EFPs. (\textbf{A}) Photomicrograph of a 40~µm thick
transverse Nissl stained section through the dorsal brainstem,
containing a superimposed, to scale, diagram of the multielectrode
probe. The probe produced a small slit in a cerebellar folium overlying
the IVth ventricle (*), and penetrated into the nucleus laminaris (NL).
The recordings were made in NL, and electrodes extended to both sides of
the nucleus. The outline of the probe is shown in light green, with the
recording electrodes indicated by magenta dots, and the reference
electrode as a magenta rectangle. The low-frequency (\textless{}1~kHz)
component (\textbf{B}) and the high-frequency (\textgreater{}2.5~kHz)
component (\textbf{C}) are ordered in the same way as the electrodes,
with three examples connected to their recording sites by black lines.
The time scales in B and C are identical (indicated by scale bar).
Traces were averaged over 10 repetitions. Voltage scales are indicated
by individual scale bars.}
\end{fullwidth}
\end{figure}

\begin{figure}[htbp]
\centering
\includegraphics[height=0.65\vsize]{figs/fig_5.pdf}
\caption{\label{fig:barnowl}The spatial structure of EFPs recorded from
the nucleus laminaris of the barn owl can be explained by a model of
axonal field potentials (for details, see Materials and Methods).
(\textbf{A}) Membrane voltage, averaged across fibers, in the model when
fit to the data. (\textbf{B}) Fitted number of fibers in the model as a
function of penetration depth. (\textbf{C}) Low-frequency ($< 1$~kHz)
components of the EFP in response to a click stimulus at time 0~ms, at
different recording depths. The depth is measured in the direction from
dorsal to ventral. Recorded responses (orange) are shown along with
model fits (black). (\textbf{D}) High-frequency ($> 2.5$~kHz)
responses in recordings (orange) and model (black). Recorded traces were
averaged over 10 repetitions.}
\end{figure}

The resulting EFP responses and the model fit are depicted in
Figure~\ref{fig:barnowl}. Figure~\ref{fig:barnowl}A shows the inferred
average over trials of the deviation of the membrane potential from the
resting potential in response to the stimulus, at a location in the axon
next to the first electrode (penetration depth 1550~µm), obtained from
the fit. The inferred voltage is composed of high- and low-frequency
components similar to those observed in the EFP. The inferred number of
fibers as a function of dorsoventral depth is shown in
Figure~\ref{fig:barnowl}B. The number (scaled by an arbitrary factor)
has its maximum at the center of the electrode array, and decays
steadily to both sides. This profile of the number of fibers is
consistent with the known anatomy of axons in NL
\citep{carr90, Kuokkanen2010Origin}.

The low-frequency (\(<1\)~kHz, Figure~\ref{fig:barnowl}C) responses
reveal the typical polarity reversal that we predicted for an axonal
terminal zone (Figures~\ref{fig:simpletree}, \ref{fig:bigtree}). The
dorsoventral depth in Figure~\ref{fig:barnowl}C and D is on the vertical
axis, which corresponds to the horizontal axis in
Figure~\ref{fig:barnowl}B. The orange lines indicate the actual
responses in the data.

The low-frequency responses at the dorsal and ventral edges in
Figure~\ref{fig:barnowl}C show the same shape, but with opposite
polarity, as expected for a dipolar field. Note that for a pure dipole
field, the amplitude of the central responses have zero amplitude. In
the data shown here, central responses show a diminished maximum
amplitude, which we interpret as the contribution of higher-order
(mostly quadrupole) components. The model is able to capture the
behaviour of this quadrupolar component as well, with a slight
underestimation of the amplitude of the peak at ventral locations. The
model even captures a small oscillation in the data with
period~of~\(\approx 1\)~ms in the center of the recording. Here, too,
the small deviations are likely due to slightly inhomogeneous conduction
velocities or non-axonal sources.

In addition to the dipolar behavior of the low-frequency response, we
also examined the high-frequency (\(> 2.5\)~kHz) response, shown in
Figure~\ref{fig:barnowl}D. The responses have a Gabor-like shape, as
expected \citep{wagner09}, with maximum amplitude in the center of the
recording array, at around 850~µm penetration depth. The axonal
conduction velocity was calculated to be 4.0~m/s, and the distance from
the bundle was 162~µm. A previously published estimate of the axonal
conduction velocity in this nucleus \citep{McColgan2014Functional} gave
a confidence bound of 0.4--6~m/s. Toward the edges (\textless{} 100~µm
and \textgreater{} 1400~µm), the amplitude decays. In the central region
(400--1200~µm recording depth), a systematic shift in delay can be
observed, while the response appears stationary in the more dorsal and
ventral electrodes. The delay increases from ventral to dorsal, which is
consistent with the anatomy for contralateral stimulation.

All these aspects of the data are qualitatively reproduced by the model
(Figure~\ref{fig:barnowl}C and D, black traces). The main deviation
between model and data lies in a diminished amplitude of the
high-frequency oscillation modelled at the most central electrode sites
(Figure~\ref{fig:barnowl}D). Because the phase shift in the central
region is mainly determined by the conduction velocity, this mismatch
might be due to a variable conduction velocity in the nucleus, and the
constant velocity in the model. \citet{McColgan2014Functional} showed
that different conduction velocities exist in the core and periphery of
the nucleus, as predicted by \citet{carr90} from variable internode
distances. A diminished amplitude in the fit could reflect an inability
of the model to exactly match the phase progression. Another possible
explanation is that the additional amplitude could be due to non-axonal
sources such as synaptic currents or postsynaptic spikes, which do not
follow the assumptions underlying our model; see the Discussion for
arguments why we expect such contributions to the EFP to be small.

When comparing the inferred membrane potential response
(Figure~\ref{fig:barnowl}A) to the measured EFP response
(Figure~\ref{fig:barnowl}C and D), the most salient difference is the
dissimilar sizes of the frequency components. In the EFP, the
low-frequency component has a comparable amplitude to the high-frequency
component, but in the membrane potential the low-frequency component is
much larger. This is because the EFP is related to membrane currents,
which are proportional to the first and second derivatives of the
membrane potential, and taking the derivative is equivalent to applying
a high-pass filter.

We performed the fitting procedure (example in Figure~\ref{fig:barnowl})
for 26 recordings from 3 different owls, with monaural stimulation from
both ears (implying the activation of distinct axonal populations). The
average correlation coefficient for all recordings was
\(R^2=0.56\pm 0.15\). The correlation coefficient for the example shown
in Figure~\ref{fig:barnowl} was 0.62.

\subsection{Dipole moments of idealized axon
bundles}\label{dipole-moments-of-idealized-axon-bundles}

We have shown theoretically and experimentally for specific examples of
axonal projection zones and inputs how dipolar EFPs emerge. We now
generalize this approach to predict the resulting dipolar EFP for
arbitrary axon and stimulus configurations. Based on our cable-theory
model, we analytically derived the maximal dipole moment
\(p_\text{max}\) for a large range of scenarios. From a given dipole
moment the maximum far field potential at distance \(r\) can be
calculated as \(\phi_\text{max}=\frac{p_\text{max}}{4\pi \sigma_e r^2}\)
where \(\sigma_e\) is the extracellular conductivity.

To simplify the analytical derivation as much as possible, we assumed a
Gaussian waveform for the membrane potential of a single spike, with an
amplitude \(\bar{V}_{\text{spike}}\) and a width
\(\sigma_{\text{spike}}\). The resting membrane potential was irrelevant
because only the first and second derivatives of the membrane potential
contribute. The axon bundle population consisted of fibers with radius
\(a\), axial resistance \(r_L\), and conduction velocity \(v\). The
population was assumed to be driven with a Gaussian firing-rate pulse
with maximum firing rate \(\bar{\lambda}_{\text{pulse}}\) and width
\(\sigma_{\text{pulse}}\). The distribution of the number of fibers at a
given depth location was also described with a Gaussian, with width
\(\sigma_n\) and maximum number \(\bar{n}\). This is an adequate
approximation if the spikes in the incoming fibers contribute little to
the dipole moment before reaching the projection zone. In this scenario,
we calculated the peak dipole moment of the bundle (see Materials and
Methods for details) to be

\begin{align}
p_\text{max} = \frac{2  \pi^2  a^2 \bar{n} \bar{\lambda}_{\text{pulse}} \bar{V}_{\text{spike}}}{\sqrt{e} r_L } \cdot
\frac{v \sigma_n \sigma_{\text{pulse}} \sigma_{\text{spike}}}
{\left(\sigma_n^2+v^2 \left(\sigma_{\text{pulse}}^2+\sigma_{\text{spike}}^2\right)\right)} \label{eqn:pmax}
\quad .
\end{align}

Equation~\ref{eqn:pmax} tells us that the dipole moment is proportional
to \(a^2\), \(\bar{n}\), \(\bar{\lambda}_{\text{pulse}}\),
\(\bar{V}_{\text{spike}}\), and \(1/r_L\). The dependence on \(v\) and
the widths is more complicated; the response is maximal with respect to
the three (spatial) widths \(\sigma_n\), \(v\sigma_{\text{pulse}}\) and
\(v\sigma_{\text{spike}}\) when they satisfy the condition
\(w_1^2=w_2^2+w_3^2\) where \(w_1\) is the largest of the three terms,
while \(w_2\) and \(w_3\) are the other two terms, regardless of order.
The dipole moment is thus maximal when the widths of the spike, the
pulse, and the terminal zone agree. In particular, if \(\sigma_n\) (the
width of the terminal zone) is the widest, then the dipole moment is
maximal if \(\sigma_n\) is equal to the spatial width of the overall
activity in the axons, which is
\(v\sqrt{\sigma_{\text{spike}}^2+\sigma_{\text{pulse}}^2}\). The widths
add in this way because the overall activity is the convolution of two
Gaussians.

Using this formula, it is then possible to calculate the expected
contributions to the EFP for different scenarios. To test the
approximation in the case of the barn owl, we chose the following
values: axon radius \(a\)~=~1~µm, conduction velocity
\(v\)~=~4~\(\frac{\text{m}}{\text{s}}\) as inferred in the previous
section, axial resistivity \(r_L\)~=~1 \(\Omega\text{m}\), and
extracellular conductivity \(\sigma_e\)~=~0.33
\(\frac{\text{S}}{\text{m}}\) as used in studies of the cortex
\citep{Gold2006Origin, Holt1999Electrical}, anatomical and physiological
parameters \(\sigma_n\)~=~500 µm, \(\bar{n}\)~=~80000,
\(\bar{V}_{\text{spike}}\)~= 70~mV from \citep{carr90}, and activation
patterns for click stimulation from \citep{koppl97a, Carr2016Role}:
\(\bar{\lambda}_\text{pulse}\)~=~1000 spikes/s,
\(\sigma_\text{spike}\)~= 250~µs, \(\sigma_\text{pulse}\)~=~0.5 ms. This
leads to a value for the dipole moment of
\(p_\text{max} \approx 2.5\, \text{µA}\cdot\text{mm}\). At a distance of
750~µm, roughly the furthest distance recorded with the multielectrode
array in Figure~\ref{fig:expmethod} and Figure~\ref{fig:barnowl}, this
dipole moment corresponded to a field potential of 1.1~mV, consistent
with the order of magnitude of the responses in our experiments
(Figures~\ref{fig:expmethod}, \ref{fig:barnowl}).

Dipole sources are also to be expected to make up the majority of the
electrical signals recorded at the scalp \citep{Nunez2006Electric}. One
such signal is the auditory brainstem response (ABR), which is recorded
at the scalp in response to auditory stimulation with clicks or chirps
\citep{Riedel2002Comparison}. An amplitude of about 10~µV of the ABR in
the barn owl has recently been reported by
\citet{PalancaCastan2016Binaural}. We calculated the contribution
expected from an axon bundle with the same characteristics as described
before at 2~cm from NL, aiming to estimate the contribution to the ABR.
Multiplying by a factor of 2 to account for the fact that there is an NL
in each hemisphere, the predicted contribution was 3.1~µV, which is of
the same order of magnitude as the value reported in the experiments.

To estimate the low-frequency dipole moment of NL from our
multielectrode recordings, it is sufficient to use CSD analysis in one
dimension, i.e.
\(\frac{\partial ^2}{\partial z^2} \phi(z) = \frac{1}{\sigma_e} i(z)\)
and a simple sinusoidal approximation of the voltage within NL:
\(\phi(z) = \phi_0 \sin(2\pi z / L)\) for \(-L/2 < z < L/2\) and
\(\phi(z) = 0\) otherwise, where \(\phi_0 \approx 0.5\)~mV is the
amplitude, \(L \approx 2\)~mm is the spatial wavelength, and \(z\) is
the depth in NL with \(z=0\) being in the center. To convert the current
density \(i\) into a current, we approximate the NL volume that
contributes to the dipole as \(V_{\text{NL}} \approx 6\)~mm\(^3\)
\citep{Kuokkanen2010Origin}. We assume that the current is homogeneously
distributed in the directions perpendicular to \(z\). Using the
definition of a dipole, \(p_{max} := \int {\rm d} V \, i(z) z\), we can
integrate over the dimensions perpendicular to \(z\) and obtain
\(p_{max} = \frac{V_{\text{NL}}}{L} \int_{-L/2}^{L/2} {\rm d} z \, i(z) z\).
Substituting \(i(z)\) and solving the integral, we find the maximum
dipole moment to be
\(p_{max} = 2 \pi V_{\text{NL}} \sigma_e \phi_0 /L \approx 3\ \text{µA}\cdot\text{mm}\),
which is consistent with our previous estimates.

As a second example, we considered thalamocortical projections, for
which \citet{Swadlow2000Influence} reported amplitudes of extracellular
spike-related potentials, called axon terminal potentials, at various
locations; for example, at 400 µm from the center of the dipole, they
reported an amplitude of the response of \(\approx\)~1~µV. Individual
thalamocortical axons are thin and have large and highly branched
projection zones \citep{Feldmeyer2012Excitatory}, so we estimated
\(\sigma_n\)~=~250~µm, \(\bar{n}\)~=~30, and \(a\)~=~1~µm. We assumed a
jitter \(\sigma_\text{pulse}\)~=~125~µs in the arrival time instead of a
true activity pulse, and we normalized the pulse to have area 1 because
we were considering a spike triggered average. The conduction velocity
has been reported as \(v\)~=~8.5~m/s \citep{Simons2007Thalamocortical}.
Leaving all other values as in the previous approximation, we arrived at
a dipole moment of \(p_\text{max} \approx 1.5 \text{µA}\cdot\text{µm}\),
yielding an extracellular spike amplitude of \(\approx\)~2.3~µV at the
distance of 400~µm, which is of the same order of magnitude as the value
(\(\approx\)~1~µV) reported by \citet{Swadlow2000Influence}.

In cases in which the jitter \(\sigma_\text{pulse}\) is longer, the
dipole moment is lower. For example, for pulses evoked by a visual
stimulus, the pulse durations can exceed 10 ms
\citep{Mitzdorf1985Current, Schroeder1991Striate, Schroeder1998Spatiotemporal, Self2013Distinct}.
Using the same parameters as for the thalamocortical projection employed
before, but increasing the number of fibers by a factor of 100,
increasing the width of the pulse to 10 ms, and increasing the maximal
firing rate to 10 Hz, we found that the value of the dipole moment was
\(p_\text{max} \approx 0.018\,\text{µA}\cdot\text{µm}\), which is two
orders of magnitude smaller than in the case of the brief pulse
discussed above. However, when we further reduced the conduction
velocity to 0.4 m/s, the same 10 ms pulse produced a dipole moment of
\(0.39\, \text{µA}\cdot\text{µm}\). Such low conduction velocities can,
for example, be found in cortico-cortical projections
\citep{Swadlow1989Efferent}.

To summarize, Equation~\ref{eqn:pmax} quantitatively predicts the
contribution of axonal projection zones to the far field EFP, and this
prediction matched experimental values in several cases.

\section{Discussion}\label{discussion}

Numerical simulations, analytical calculations, and experimental data
allow us to show how axonal fiber bundles may contribute to the EFP, and
explain how the contributions are shaped by axonal morphology. There are
three principal effects of axon bundle structure on the EFP. First, the
low-frequency components of the EFP are governed by the densities of
bifurcations and terminations and can have a dipolar structure
(Figure~\ref{fig:simpletree} and Figure~\ref{fig:bigtree}A-C). Second,
the high-frequency components are governed by the local number of fibers
(Figure~\ref{fig:bigtree}D-F). Third, membrane potentials that change on
the same spatial scale as an axonal projection zone through which they
propagate generate strong dipole moments in the EFP response
(Figure~\ref{fig:distscaling}). At the temporal frequencies that
correspond to wavelengths of the size of the projection zone, this leads
to dipolar EFP components that are not negligible and exceed the reach
of the presumed quadrupolar nature of axonal EFPs.

\subsection{Relevance to the interpretation of electrophysiological
recordings}\label{relevance-to-the-interpretation-of-electrophysiological-recordings}

Our findings relate to the interpretation of a wide range of
electrophysiological data in general, and to the estimation of current
sources in particular. When performing a typical current source density
(CSD) analysis, the local number of fibers cannot be disentangled from
membrane current density
\citep{Nicholson1973Theoretical, Potworowski2011Kernel}. In CSD
analysis, the membrane current densities can vary independently with
time and location. In contrast, in the case of an axonal fiber bundle
considered here, the situation is different: the number of fibers is
variable in space, in particular in the terminal zone, but the current
sources at different locations are highly correlated because they are
caused by propagating action potentials. In the case presented here
(Figure~\ref{fig:barnowl}) where axonal action potentials dominate the
EFP, it was possible to recover the (normalized) fiber densities and
average membrane potentials from the recordings.

Beyond recovering actual fiber densities and membrane potentials, our
approach enables the interpretation of CSD results in the presence of
axon fiber bundles. For example, the sink and source distribution found
in classical CSD analysis of axon bundles
\citep{Mitzdorf1978Prominent, Mitzdorf1985Current, Mitzdorf1977Laminar}
shows a dipolar structure in terminal zones, but a conclusive
explanation of their origin was not given.
\citet{Tenke1993Interpretation} studied the dipole at an axonal terminal
zone in the macaque striate cortex for a fixed point in time,
attributing the sinks to the depolarized axon endings, and the sources
to the return currents distributed along the axons, while not taking
account of additional currents flowing at bifurcations. Our modeling
approach provides a novel way of interpreting these findings in terms of
actively propagated action potentials in a fiber bundle.

As an example case for a fiber bundle, we recorded from the barn owl
nucleus laminaris. Figure~\ref{fig:expmethod} and
Figure~\ref{fig:barnowl} showed that the low- (\textless{}1~kHz) and
high-frequency (\textgreater{}2.5~kHz) components exhibit qualitatively
different behaviours as a function of recording location relative to the
terminal zone. The low-frequency component is a largely stationary
phenomenon, while the fine structure of the high-frequency component
shifts gradually in space as a function of the axonal conduction
velocity (Figure~\ref{fig:barnowl}, see also \citet{Carr2015Maps}).
Low-frequency components have a strong dipole moment, meaning that they
contribute to the far-field EFP. Due to the difference in reach, the
high-frequency component is most suitable for the study of local
phenomena while the low-frequency component bears information about
locations more distant from the recording site
(Figure~\ref{fig:distscaling}), consistent with findings on non-axonal
EFPs \citep{Pettersen2008Amplitude, Leski2013Frequency}.

Note that the low-pass (\textless{} 1~kHz) filtered EFP is calculated in
a similar way to the LFP, with the exception that the cutoff frequencies
used to separate the low- and high-pass filtered EFP are relatively high
compared to those used in cortical or hippocampal studies to separate
LFP and MUA. We applied these high cut-offs because our modeling and
experiments were performed in the auditory brainstem of the barn owl,
which operates on very short time scales and, consequently, higher
frequencies. We expect other systems operating on slower time scales to
have lower optimal cutoff frequencies separating the components.
Equation~\ref{eqn:pmax} indicates how the different spatial and temporal
system properties relate to each other to generate a dipole moment.

Dipolar fields are essential for the generation of electrical field
potentials at greater distances from the brain. The most prominent of
these is the EEG, which is commonly attributed to the dipolar
contributions of pyramidal cells \citep{Nunez2006Electric}. As
originally suggested by \citet{Tenke1993Interpretation}, we propose that
axonal contributions might also be relevant in the analysis of these
fields. This is particularly true for the auditory brainstem response
(ABR), which is closely related to the EEG and involves brain structures
that display high degrees of synchrony as well as axonal organization,
and are thus ideal candidates for the generation of axonal field
potentials visible at long ranges. This would in turn have implications
for the interpretation of the ABR in clinical contexts.

The ABR amplitude of the barn owl has been reported to be on the order
of 10 µV \citep{PalancaCastan2016Binaural} while we estimated a
contribution of about 3 µV amplitude from the incoming axons in NL
alone. This estimate of 30\% axonal contribution to the ABR suggests
that there may indeed be measurable components due to axons in the ABR,
in particular, and the EEG, in general. However, this estimate is crude
because it did not take into account the anatomy of the skull except for
its size. Future studies based on a more detailed skull model and paired
recordings of ABR and EFP should improve our understanding of axonal
contributions to the ABR.

We have shown that the EFP in the barn owl NL is consistent with a model
of axonal sources. We believe synaptic contributions to be small in this
case for the following reasons: The somatic membrane potentials due to
synaptic currents are much smaller than the impact of postsynaptic
spikes \citep{Ashida2007Passive, Funabiki2011Computation}. Since
postsynaptic spikes contribute little to the EFP
\citep{Kuokkanen2010Origin}, we suspect that the synaptic contributions
to the EFP are also small. Furthermore, synaptic EFP contributions would
require a spatial separation of currents, which is not possible to
achieve in NL because of the symmetrical arrangement of synapses on the
spherical NL cell bodies \citep{carr90}, meaning that synaptic sources
can also not explain a dipolar EFP, and are thus expected to contribute
little to the EFP.

\subsection{Dipolar EFPs in other animals and brain
regions}\label{dipolar-efps-in-other-animals-and-brain-regions}

It is interesting to note that a similar dipole-like reversal of
polarity as shown here for the barn owl NL has been reported in the
chicken NL \citep{Schwarz1992Can} as well as in the mammalian analog to
NL, the medial superior olive (MSO) \citep{McLaughlin2010Oscillatory}.
The phase reversal in this case was modeled based on the assumption that
the postsynaptic NL and MSO dendrites with their bipolar morphology
generate the dipolar EFP response
\citep{McLaughlin2010Oscillatory, Goldwyn2014Model}. However, in the owl
this dipolar morphology of neurons is largely absent \citep{carr90},
making dendritic sources unlikely. This differential morphology suggests
that similar dipolar field potentials in owls and mammals emerge from
different physiological substrates. Such a convergence might point
towards an evolutionary pressure favoring a bipolar EFP structure in
coincidence detection systems, and indeed, \citet{Goldwyn2016Neuronal}
have proposed a model in which this extracellular potential enhances the
function of coincidence detectors through a form of ephaptic
interaction. Their approach centers on dendrites and is not directly
transferable, but it seems possible that a similar mechanism might arise
in the barn owl NL based on axonal EFPs.

The key assumption underlying our modeling of axonal geometries is that
there exists a preferential direction of the axon arbor. In many
structures this is the case, for example in the parts of the auditory
brainstem we studied here. In other brain regions, this tendency is not
as prominent, with a spectrum existing between completely directed and
undirected growth. More undirected growth would lead to a more diffuse
response in the EFP, and eventually to a cancellation of the dipolar
field potentials. \citet{Cuntz2010One} and \citet{Budd2010Neocortical}
studied the principles underlying the growth patterns of axons and found
that the degree of direction in the growth of an axon depends on the
balance struck between conduction delay and wiring cost. Optimizing for
minimal conduction time leads to highly directed structures while
optimizing wiring cost leads to more tortuous, undirected growth. This
insight suggests that directed structures - and thus also strong,
dipolar EFPs due to axons - may be more prevalent in systems which
require high temporal precision in the information processing. This
requirement for high temporal precision also aligns well with our model
prediction: the dipole moment (Equation~\ref{eqn:pmax})
is maximal when the spatial spread of the activation is equal to the
size of the projection zone, favoring short activation times
(\textless{} 1 ms) for typical projection zone sizes (1 mm) and
conduction velocities (1 m/s).

\subsection{Relationships to more detailed biophysical
models}\label{relationships-to-more-detailed-biophysical-models}

In the systems we were aiming to describe with our model, for example NL
and thalamocortical projections, synaptic boutons are typically small,
and we did not model them explicitly. In other systems such as the
neuromuscular junction, the synaptic structure can be very large when
compared to the axon bundle
\citep{Harris1979Relationship, Katz1961Terminations, Katz1965Propagation}.
Such a large junction with an overall length of up to 1~mm was modeled
by \citet{Gydikov1986Extracellular}. They found a significant effect of
this structure on the EFP. The single flaring and tapering neuromuscular
junction in their model had a comparable effect as the entire projection
zone in our model, with the flaring causing a similar effect as the
bifurcations, and the tapering taking the role of the terminations in
our model. Given that synaptic boutons are several orders of magnitude
smaller in NL and cortex, we do not expect a strong effect in these
systems.

Membrane currents flowing in boutons were studied by
\citet{Geiger2000Dynamic}, who recorded from the terminals of
hippocampal mossy fibers and examined calcium and potassium
conductances. The potassium conductances broadened the incoming spikes
in an activity-dependent manner. This spike broadening is hypothesised
to be mediated by slow inactivation of the potassium channels and takes
place on a timescale of \textgreater{} 100 ms, and is thus not relevant
to the present study. Spike broadening could be captured in our model by
incorporating in Equation~\ref{eqn:pmax} a
\(\sigma_\text{spike}\) that is variable in time.

The calcium currents reported by \citet{Geiger2000Dynamic} were further
quantified by \citet{Alle2009EnergyEfficient}. Calcium currents were
temporally overlapping and much smaller in amplitude than sodium and
potassium currents. We therefore neglected calcium currents in our
model.

Modeling the myelinated compartments, we assumed that they are purely
passive and strongly insulated from the extracellular space. However,
myelinated compartments do in fact express active conductances, in
particular in the paranodal and juxtaparanodal region
\citep{Chiu1981Evidence, Waxman1985Organization}. Including such a
detailed distribution of ion channels in our model could lead to a
different shape of the waveform and the spectrum of the EFP of an action
potential, possibly similar to the effect described by
\citet{Ness2016Active} for active conductances on dendrites. The
conclusions drawn by our model are, however, independent of the precise
active conductances and the distribution of myelinated and active
segments along the axons because our results rely only on the gross
waveform of propagating action potentials, but not on finer details.
Active conductances and capacitive currents in the myelinated segments
could affect the shape of the action potential waveform, but do not
affect our conclusion about the spatial scaling behaviour of the EFP.

Because of the weak dependence of our results on the gross extracellular
spike waveform, our analytical model does not include any intrinsic
low-pass filtering as can be derived, for example, for dendritic models
\citetext{\citealp{Linden2010Intrinsic}; \citealp{Einevoll2013Modelling}; \citealp[for
reviews see][]{Buzsaki2012Origin}}. The effective additional currents
flowing at bifurcations and terminations are, however, low-frequency
contributions to the overall membrane currents in our model. Extending
our model to treat these currents separately might show whether axons
could contribute to the observed \(1/f\) scaling of the spectrum of the
EFP \citep{Pritchard1992Brain}.

\subsection{Conclusion}\label{conclusion}

Axonal projections can contribute substantially to EFPs. Our results
quantitatively show how the anatomy of axon terminal zones and the
activity in axons determine its frequency-specific far-field
contribution to the EFP.

\section{Materials and Methods}\label{materials-and-methods}

\subsection{Experimental recordings}\label{experimental-recordings}

The experiments were conducted at the Department of Biology of the
University of Maryland. Data was collected from three barn owls
(\emph{Tyto furcata pratincola}). Procedures conformed to NIH Guidelines
for Animal Research and were approved by the Animal Care and Use
Committee of the University of Maryland. Anaesthesia was induced prior
to each experiment by intramuscular injection of a total of \(8-10\)
ml/kg of \(20\%\) urethane divided into three to four injections over
the course of \(3\) hours. Body temperature was maintained at
\(39^\circ\)C by a feedback-controlled heating blanket.

Recordings were made in a sound-attenuating chamber (Acoustic Systems
Inc., Austin, TX, USA). Tungsten electrodes with impedances between
\(2\) and \(20\) M\(\Omega\) were used to find suitable recording
locations in nucleus laminaris (NL). Once NL had been located, the
tungsten electrode was retracted and replaced with a 32 channel
multi-electrode array (A1x32-15mm-50-413-A32, Neuronexus, Ann Arbor, MI,
USA). The multi-electrode array was lowered using a microdrive (MP225,
Sutter Instruments Co., Novato, CA, USA) during continuous presentation
of a white-noise burst stimulus until visual inspection of the waveform
showed that NL was at the center of the array. A grounded silver
chloride pellet, placed under the animal's skin around the incision,
served as the animal ground electrode. Electrode signals were amplified
by a headstage (HS36, Neuralynx Inc., Tucson, AZ, USA). An adapter
(ADPT-HS36-N2T-32A, Neuralynx Inc.) was used between the electrode and
the headstage. A further adapter (ADPT-HS-36-ERP-27, Neuralynx Inc.) was
used between the headstage and the control panel in order to map all 32
channels to the amplifiers. Pre-amplified electrode signals were passed
to the control panel (ERP27, Neuralynx Inc.), then to four 8-channel
amplifiers (Lynx-8, Neuralynx Inc.), and then to an analogue-to-digital
converter (Cheetah Digital Interface, Neuralynx Inc.) connected to a
personal computer running Cheetah5 software (Neuralynx Inc.) where the
responses were stored for off-line analysis.

Acoustic stimuli were digitally generated by a custom-made
\textsc{Matlab} (MathWorks, Natick, MA, USA; RRID:SCR\_001622) script \citep{GithubRepo} driving a
signal-processing board (RX6, Tucker-Davis Technologies, Alachua, FL,
USA) at a sampling rate of 195.3125~kHz. The sound stimuli were
attenuated using a programmable attenuator (PA5, Tucker-Davis
Technologies). Click stimuli were generated as a single half-wave of a
5~kHz sine tone. Miniature earphones were inserted into the owl's ear
canals and fixed to a headplate. Acoustic stimuli were fed to these
earphones. Stimulus delivery was triggered by National Instruments
equipment (NI USB-6259 and BNC-2090A, National Instruments Inc, Austin,
TX, USA), and stimulus presentation times were recorded along with the
responses. Trigger pulses were configured in \textsc{MatLab} through
Ephus software (Vidrio Technologies LLC, Ashburn, VA, USA). Responses
were averaged over 10 repetitions.

\subsection{Multi-compartment model of
axons}\label{multi-compartment-model-of-axons}

We modeled axons using NEURON \citep{Hines1997NEURON, Hines2009NEURON}
and extended previous work by \citet{Simon1999Dendritic} and
\citet{Kuba2009Roles}, which included the high- and low-threshold
potassium channels used by \citet{Rathouz1998Characterization}. The axon
was modeled as a sequence of active nodes and passive myelinated
segments. The nodes and myelinated segments had lengths of 2~µm and
75~µm, respectively. We used the model of a nucleus magnocellularis (NM)
axon provided by \citet{Simon1999Dendritic} in ModelDB
(\citet{Hines2004ModelDB}, Accession number: 153998). In order to ensure
robust spike propagation at the bifurcations, some of the model the
parameters were modified. The values of the properties that were
modified from those provided by \citet{Simon1999Dendritic} are shown in
Table 1. In addition, the Q10 values were set to 3, and the temperature
was set to \(40^{\circ}\,\text{C}\) as done by \citet{Kuba2009Roles}. The
ratio of leak conductance and capacitance between node and myelin was
changed from 80 as used by \citet{Simon1999Dendritic} to 1000
\citep{Koch2004Biophysics}. We removed the Hodgkin-Huxley type potassium
conductivities included by \citet{Simon1999Dendritic} (which are based
on data from the squid axon) from the simulations, and increased the
KLVA and KHVA conductances (which are based on physiological data from
the auditory brainstem) to compensate. In order to avoid nodes lining up
in axial direction, the very first myelinated segment had a length drawn
from a uniform distribution between 0 and 75~µm.

We included branching axons in our simulations. Branches were generated
by connecting the incoming passive segment to one end of a node, and the
two outgoing passive segments to the other end of the node, and then
continuing the alternation of active and passive segments in each
resulting branch. In Figure~\ref{fig:simpletree}A-D, where the positions
of bifurcations or terminations of axons were fixed, the last node was
placed in the axon as usual, and the last myelinated segment was
shortened in order to obtain the total length before the bifurcation or
termination.

To approximate the infinitely long axons, we added segments before and
after the shown portions of the axon. We chose the total length of the
additional segments by incrementally increasing the length
segment-by-segment until there was no visual difference between each
successive lengthening of the axon, arriving at a length of 3 mm at each
end.

To evoke an action potential at a designated time, a special conductance
was temporarily activated in the first node of Ranvier. The conductance
had a reversal potential of 0 mV, a maximal amplitude of 0.05~µS, and a
time course described by an alpha-function with time constant 0.01~ms.
Soma and axon initial segment were not modeled explicitly. This
conductance resembled a synaptic conductance, except for the very short
time constant and the lack of a somatic or dendritic compartment on
which synapses typically impinge.

For the simplified axon geometries used in Figure~\ref{fig:simpletree},
the branching pattern was fixed as described in the caption, with the
exception of the axial positions of the branching points in
Figure~\ref{fig:simpletree}E: a random offset between branching points
was drawn from a gamma distribution with mean 400~µm and standard
deviation 300~µm. The initial branching point for each axon was offset
from the original location by a distance drawn from a Gaussian
distribution with mean zero and a width of 300 µm. This was done to
smooth out the effects of individual branchings or terminations.

For the axons in Figure~\ref{fig:bigtree}, branching patterns were
generated procedurally, starting with a root segment. In order to avoid
artifacts from the stimulus and to simulate a long fiber tract prior to
the terminal zone, a sequence of 10 active and passive segments without
bifurcations was assumed before the terminal zone (770~µm total length).
To this root, segments were appended iteratively. Before adding a
segment, a decision whether to branch or terminate an axon was drawn
from a probability distribution that was dependent on the axial position
of the end of the previous segment. These probability distributions were
modeled as logistic functions with the parameters adjusted to roughly
match the numbers of branchings and terminations shown by
\citet{carr90}. Thus, an initial phase dominated by bifurcations was
followed by a phase dominated by terminations, with the probability of
termination reaching 100\% at the end of the terminal zone. The
distribution of bifurcations had its maximum at axial location \(z=0\)
with a standard deviation of 200 µm. The distribution of terminations
had its maximum at \(z=500\) µm, with a standard deviation of 100 µm.
The branching angle had an average of \(20^\circ\), with a standard
deviation of \(5^\circ\). At branching points, the plane containing the
branches had a uniform random orientation, resulting in a 3-dimensional
structure of the axon bundle.

In all cases except for the simulations shown in
Figure~\ref{fig:bigtree}, the radial position of all node and myelin
compartments was set to zero, meaning they were placed on a straight
line extending axially.

Numerical simulations of action potentials propagating along axons
yielded the membrane currents from which we calculated extracellular
fields. This procedure is described in detail by \citet{Gold2006Origin}.
Briefly, the extracellular medium is assumed to be a homogeneous volume
conductor with conductivity \(\sigma_e\), and a quasi-static
approximation of the electrical field potential \(\phi\) is made. The
extracellular potential \(\phi(\mathbf{r},t)\) at location
\(\mathbf{r}\) and time \(t\) due to a membrane current density
distribution \(i(\mathbf{r},t)\) is then governed by the equation
\(\Delta \phi(\mathbf{r},t) = \frac{1}{\sigma_e} i(\mathbf{r},t)\), with
\(\Delta\) denoting the Laplace operator \citep{Nunez2006Electric}. If
the currents \(i\) are constrained to a volume \(W\), this equation has
the solution

\begin{equation}
\phi(\mathbf{r},t)=\frac{1}{4\pi\sigma_{e}}\int_{W}\frac{i(\mathbf{r}',t)}{|\mathbf{r}-\mathbf{r}'|}\textrm{d}\mathbf{r}' \ .
\label{eqn:basic}
\end{equation}

Since the majority of the current flows through the small nodes of
Ranvier in myelinated axons, we used the point-source approximation for
all compartments, and subdivided the myelinated segments into 10
iso-potential sections each; we did not use the line-source
approximation \citep{Holt1999Electrical}.

Analysis of the resulting extracellular field potential (EFP) included
filtering. All filtering was performed with third-order Butterworth
filters. The multi unit activity (MUA) was calculated by high-pass
filtering the signal with a cutoff of 2500~Hz, setting all samples with
negative values to zero, and then low-pass filtering the resulting
response with a cutoff of 500~Hz. The low-pass filtered EFP was
calculated with a cutoff of 1000~Hz. To exclude influences of spectral
leakage on our results, we also performed the same analysis with
20th-order Butterworth filters and found qualitatively identical
results. The specific filtering (3rd or 20th order Butterworth) did not
affect our conclusions.

The code for these simulations is available at
\url{https://github.com/phreeza/pyLaminaris} \citep{GithubRepo}.

\begin{table}[bt]
\centering
\caption{Parameter values used for the multi-compartment model which
were modified from those used by \citet{Simon1999Dendritic}.
}
  \begin{tabular}[t]{l p{5cm} l p{4cm}}
\toprule
    Symbol & Meaning & Value & \raggedright{Value used by \\
    \citet{Simon1999Dendritic}}\tabularnewline
\midrule
\(R_a\) & axial resistance & 50 \(\Omega\) cm & 200 \(\Omega\)
cm\tabularnewline
    \(\bar{g}_\text{Na}\) & \raggedright{maximum \\ sodium conductance} & 2.4
\(\text{S}/\text{cm}^2\) & 0.32 \(\text{S}/\text{cm}^2\)\tabularnewline
    \(\bar{g}_\text{KLVA}\) & \raggedright{maximum low-threshold \\ potassium conductance} &
0.1 \(\text{S}/\text{cm}^2\) & 3
\(\text{mS}/\text{cm}^2\)\tabularnewline
    \(\bar{g}_\text{KHVA}\) & \raggedright{maximum high-threshold \\ potassium conductance} &
1.5 \(\text{S}/\text{cm}^2\) & 30
\(\text{mS}/\text{cm}^2\)\tabularnewline
\(g_\text{leak}^\text{node}\) & leak conductance in node & 1
\(\text{mS}/\text{cm}^2\) & 0.28
\(\text{mS}/\text{cm}^2\)\tabularnewline
\(g_\text{leak}^\text{myelin}\) & leak conductance in myelin & 1
\textmu\(\text{S}/\text{cm}^2\) & 35
\textmu\(\text{S}/\text{cm}^2\)\tabularnewline
\(E_\text{leak}\) & leak reversal potential & -72 mV & -45
mV\tabularnewline
\(E_\text{K}\) & potassium reversal potential & -80 mV & -60
mV\tabularnewline
\(E_\text{Na}\) & sodium reversal potential & 50 mV & 40
mV\tabularnewline
\(c_m^\text{myelin}\) & \raggedright{membrane capacitance\\ in myelin} & 1
\(\text{nF}/\text{cm}^2\) & 12 \(\text{nF}/\text{cm}^2\)\\
\bottomrule
\end{tabular}
\end{table}

\subsection{Mean-field model of an axon
bundle}\label{mean-field-model-of-an-axon-bundle}

\label{sec:efpresp}

To better understand the processes leading to the complex
spatio-temporal patterns of extracellular fields, we devised an
analytically tractable model of axon bundles. We defined the spatial
dimension in cylindrical coordinates \(\mathbf{r}=(\rho,\varphi,z)\),
where \(\rho\) was the radial distance from the cylindrical axis,
\(\varphi\) the angle of azimuth, and \(z\) the axial distance along the
cylinder axis. We considered a simple model axon bundle that extended
infinitely in the axial \(z\)-direction at \(\rho=0\). The bundle had a
variable number of fibers along the \(z\) axis, denoted by \(n(z)\),
each of which cylindrical with an identical radius \(a\). This meant
that the total cross-sectional area \(A\) of the bundle at a given depth
\(z\) was given by \(A(z)=\pi a^2 n(z)\). We assumed the axons to be
perfect transmission lines, meaning that the action potential is a
traveling wave with velocity \(v\) along the axon. In particular, we
neglected delays and distortions that can be induced when an axon
branches or terminates. In this case, we could assume that the membrane
voltage was the same in each fiber for a given \(z\) coordinate. From
linear cable theory \citep[e.g.][]{Jack75Electric}, we then obtained the
following expression for the total transmembrane current per unit length
\(I(z,t)\) from a given membrane potential \(V(z,t)\):

\begin{align}
I(z,t)& = \frac{\partial }{\partial z}\left(\frac{A(z)}{r_L}\frac{\partial }{\partial z}V(z,t)\right)\\
&= \frac{\pi a^2}{r_L}\frac{\partial }{\partial z}\left(n(z)\frac{\partial }{\partial z}V(z,t)\right) \\
&= \frac{\pi a^2}{r_L}\left(\frac{\partial}{\partial z}n(z)\cdot\frac{\partial}{\partial z}V(z,t)+n(z)\cdot\frac{\partial ^2}{\partial z^2}V(z,t)\right)
\label{eqn:current}
\end{align}

We next calculated the corresponding extracellular field potential
\(\phi(\mathbf{r},t)\) of a given membrane potential waveform \(V(z,t)\)
propagating through the axon bundle. Due to the rotational symmetry of
the system and the fact that current flows only at \(\rho = 0\), the
membrane current can be described as
\(i(\mathbf{r},t)=I(z,t)\frac{\delta(\rho)}{\rho}\), where
\(\frac{\delta(\rho)}{\rho}\) is the Dirac delta distribution for a line
at \(\rho=0\). Applying this membrane current to Equation~\ref{eqn:basic}, we obtained

\begin{align}\label{eqn:simple_field_pot}
    \phi(\mathbf{r},t) =\frac{1}{4\pi\sigma_{e}}\int_{-\infty}^{\infty}\frac{I(z',t)}{\sqrt{(z-z')^2 + \rho^2}}\textrm{d}z' \ ,
\end{align}

which is independent of the angle \(\varphi\).

To derive the frequency responses in Figure~\ref{fig:distscaling}, we
simulated the membrane potentials as sinusoids, i.e.
\(V(z,t) = \sin\left(2\pi f\cdot\left(z-tv\right)\right)\), with varying
frequencies \(f\) between 100 Hz and 5~kHz and calculated the standard
deviation of the response for each frequency individually. The amplitude
of the membrane potential \(V\) was the same for all frequencies.

To derive the dipole moment of a simplified projection zone, we
considered an axon bundle in which identical spikes with the waveform
\(V_\text{spike}(z,t)\) propagate as traveling waves with a velocity
\(v\) in positive \(z\) direction:
\(V_\text{spike}(z,t) = V_\text{spike}(z-tv,0)\). If each of the fibers
is stimulated with an inhomogeneous Poisson process, with the driving
rate \(\lambda(t)\) shared among all axons, the average membrane
potential across fibers will be
\(V(z,t) = V_\text{spike}(z,t)\ast \lambda(t)\), where \(\ast\) denotes
the convolution with respect to the time \(t\). Plugging this into
Equation~\ref{eqn:current}, we obtained

\begin{align}
I(z,t)& = \frac{\pi a^2}{r_L}\left(\frac{\partial}{\partial z}n(z)\cdot\frac{\partial}{\partial z}\left[V_\text{spike}(z,t)\ast \lambda(t)\right]+n(z)\cdot\frac{\partial ^2}{\partial z^2}\left[V_\text{spike}(z,t)\ast \lambda(t)\right] \right) \\ 
& = \frac{\pi a^2}{r_L}\lambda(t)\ast\left(\frac{\partial}{\partial z}n(z)\cdot\frac{\partial}{\partial z}V_\text{spike}(z,t)+n(z)\cdot\frac{\partial ^2}{\partial z^2}V_\text{spike}(z,t) \right) \ .
\end{align}

\subsection{Analytical solution of the mean-field model of an axon
bundle}\label{analytical-solution-of-the-mean-field-model-of-an-axon-bundle}

Assuming Gaussian shapes for the firing-rate pulse
\(\lambda(t) = \bar\lambda_{\text{pulse}} \exp\left(-\frac{t^2}{2\sigma_\text{pulse}^2}\right)\),
the spike
\(V_\text{spike}(z,t) = V_\text{spike}(z-tv,0) = \bar{V}_\text{spike} \exp\left(-\frac{(z-tv)^2}{2\sigma_\text{spike}^2}\right)\),
and the fiber bundle projection zone
\(n(z) = \bar{n} \exp\left(-\frac{z^2}{2\sigma_\text{n}^2}\right)\), we
were able to take advantage of the fact that the product and the
convolution of two Gaussians are again Gaussian, and obtained

\begin{align}\label{eqn:simple_dipole_cur} I(z,t)&=\bar{n} \bar{\lambda
}_{\text{pulse}} \bar{V}_{\text{spike}}  \sqrt{2} \pi ^{3/2} a^2  \cdot \exp
\left(-\frac{z^2}{2 \sigma _n^2}-\frac{(z-t v)^2}{2 v^2 \left(\sigma
_{\text{pulse}}^2+\sigma _{\text{spike}}^2\right)}\right) \\ \nonumber
&\cdot\frac{ \sigma _n^2 \left(-v^2 \sigma _{\text{pulse}}^2-v^2 \sigma
_{\text{spike}}^2+(z-t v)^2\right)-v^2 z \left(\sigma _{\text{pulse}}^2+\sigma
_{\text{spike}}^2\right) (t v-z)}{v^4 r_L \sigma _n^2 \sqrt{\frac{1}{\sigma
_{\text{pulse}}^2}+\frac{1}{\sigma _{\text{spike}}^2}} \left(\sigma
_{\text{pulse}}^2+\sigma _{\text{spike}}^2\right){}^2} \quad .
\end{align}

The dipole moment \(p(t)\) is defined as

\begin{equation}
p(t) = \int_{-\infty}^{\infty}z\cdot
I(z,t)\text{d}z \quad ,
\end{equation}

into which we can enter Equation~\ref{eqn:simple_dipole_cur} to obtain

\begin{equation}
p(t) =- \bar{n} \bar{\lambda }_{\text{pulse}}
\bar{V}_{\text{spike}}\frac{2 \pi ^2 a^2}{r_L }\frac{v^2 \sigma_n
\sigma_{\text{pulse}} \sigma_{\text{spike}}}{\left(\sigma _n^2+v^2 \left(\sigma
_{\text{pulse}}^2+\sigma _{\text{spike}}^2\right)\right)^{3/2}}\cdot t
\exp\left(\frac{-t^2 v^2}{2 \left(\sigma _n^2+v^2 \left(\sigma _{\text{pulse}}^2+\sigma
_{\text{spike}}^2\right)\right)}\right) \quad .
\end{equation}

The dipole moment has its peak amplitude at
\(t_\text{max}=-\sqrt{\sigma_n^2+v^2 \left(\sigma _{\text{pulse}}^2+ \sigma _{\text{spike}}^2\right)}/v\),
and takes the value

\begin{equation}
p_\text{max} = p(t_\text{max}) = \frac{2 a^2
\pi^2}{r_L\sqrt{e}} \cdot \frac{v \sigma_n \sigma_{\text{pulse}}
\sigma_{\text{spike}} \bar{n} \bar{\lambda}_{\text{pulse}}
\bar{V}_{\text{spike}}} {\left(\sigma_n^2+v^2
\left(\sigma_{\text{pulse}}^2+\sigma_{\text{spike}}^2\right)\right)} \quad ,
\end{equation}

which is presented as Equation~\ref{eqn:pmax} in the Results section.

\subsection{Model fitting to experimental
data}\label{model-fitting-to-experimental-data}

In order to relate the model to experimentally obtained data as shown in
Figure~\ref{fig:barnowl}, we performed a nonlinear least squares fit,
minimizing the mean squared error \(\epsilon\) between the measured
potential \(\phi_\text{measured}\) and the model prediction
\(\phi_\text{model}\) in Equations~\ref{eqn:simple_field_pot} and
\ref{eqn:current} for \({N=32}\) measurement locations
\(z_n\)~(\(n=1,\dots,N\)) and \({M=600}\) time points \(t_m\)
(\(m = 1,\dots,M\)):
\(\epsilon = \frac{1}{NM}\sum_{n=1}^{N}\sum_{m=1}^{M}\left[\phi_\text{measured}(z_n,t_m)-\phi_\text{model}(z_n,t_m)\right]^2\).
The separation between electrodes was given by the electrode layout as
50~µm. The time between sampling points was 5.12~µs. We achieved the
minimization of the error \(\epsilon\) using the ``optimize.minimize''
routine provided by the \textsc{SciPy} package \citep{scipy}. The free
parameters to be determined by the optimization routine were the
distance \(\rho\), the velocity \(v\), the number of fibers per unit
length \(n(z_n)\) for each measurement location~\(z_n\), and the spatial
derivative of the average membrane potential
\(\frac{\text{d}}{\text{d}z}V(z_1,t_m)\) at electrode location \(z_1\)
for each time point~\(t_m\). We fit the first derivative of the membrane
potential in order to better capture the low-frequency components that
we found in Figure~\ref{fig:simpletree} E, and because the membrane
potential appears only as the derivative in the model. The derivative of
the membrane voltage at the other locations than \(z_1\) was then
determined by the traveling-wave assumption:
\(\frac{\text{d}}{\text{d}z}V(z_n,t_m) = \frac{\text{d}}{\text{d}z}V(z_1,t_m-\frac{z_n-z_1}{v})\),
using a linear interpolation between timepoints. The model assumption of
a single line of axons with electrodes at a fixed distance is a
simplification of a three-dimensional axon tree where the fibers are
distributed at various distances. The distance parameter \(\rho\) in
Equation~\ref{eqn:simple_field_pot} can be interpreted as an average
distance in this simplification.

To aid the convergence of the fit algorithm, an initial guess for the
number of fibers \(n(z_n)\) was set by hand. Initializing the guess to a
constant or a fully random number of fibers resulted in a failure to
converge. However, different Gaussian-like initial guesses converged to
a single solution, meaning that the specific initial guess did not alter
the final fit result. Initializing the membrane voltage with different
normally distributed values did not affect the outcome of the fit. The
results shown in Figure~\ref{fig:barnowl} were obtained with an initial
guess of a Gaussian with amplitude 12, centered at penetration depth
725~µm with standard deviation 400~µm.

Because of the linearity of Equations~\ref{eqn:basic}-\ref{eqn:simple_field_pot} both in the current \(I\) and
the membrane potential \(V\), inferring the membrane voltage \(V\) from
the average over trials of the extracellular potential \(\phi\) produces
the average membrane voltage \(V\). This in turn is the membrane voltage
response of a single spike convolved with the peri-stimulus time
histogram (PSTH).

\section{Acknowledgements}\label{acknowledgements}

This work was supported by the German Federal Ministry for Education and
Research Grants 01GQ1001A and 01GQ0972, NIH DCD 000436 and US-American
Collaboration in Computational Neuroscience ``Field Potentials in the
Auditory System'' as part of the NSF/NIH/ANR/BMBF/BSF Collaborative
Research in Computational Neuroscience Program, 01GQ1505A.

The authors wish to thank Anna Kraemer for help with animal handling and
surgery; and Martina Michalikova and Tiziano D'Albis for helpful
comments on a draft of this manuscript.

\renewcommand\refname{Bibliography}
\bibliography{doc/refs.bib}

\end{document}
